\hypertarget{como-se-obtuvieron-y-se-estructuraron-los-datos-para-cada-producto}{%
\subsubsection{¿Como se obtuvieron y se estructuraron los datos para
cada
producto?}\label{como-se-obtuvieron-y-se-estructuraron-los-datos-para-cada-producto}}

Para obtener datos de interés sobre cada producto, se visitó
personalmente el establecimiento donde dicho producto podía ser
encontrado, y se tomaron las métricas necesarias para desarrollar los
análisis que podían resultar interesantes. Desde el precio medio de la
pizza en Plaza hasta si importa o no que varios establecimientos estén
cercanos entre sí para conformar un precio, son algunas de las preguntas
que podrán ser contestadas tomando como base los datos encontrados.

\hypertarget{-cerveza}{%
\paragraph{-Cerveza:}\label{-cerveza}}

En cuanto a la cerveza, los datos analizados fueron: marca, precio, tipo
de envase, color del envase, contenido, grado de alcohol, país de origen
y ubicación de cada instancia del líquido. Haciendo uso de dichos datos
para cada cerveza, se pueden establecer relaciones que sean un indicador
de la variedad, costo y adaptabilidad del precio según el lugar donde
está situado el establecimiento y otras variables. A partir de dichas
preguntas, se pueden sacar conclusiones respecto a la capacidad del país
para competir con el producto importado.

\hypertarget{-pizza}{%
\paragraph{-Pizza:}\label{-pizza}}

Sobre la pizza se tomaron los datos: tamaño, agregados, precios de los
agregados, tipos especiales de pizza y cercanía a lugares de interés y
de establecimientos que vendían pizza entre sí. A partir de estos datos,
es posible conocer los costos que tiene este producto y la capacidad de
compra de la moneda nacional para conformar un almuerzo que, a priori,
no parece algo que debería ser preocupante, por lo que se pueden validar
o desmentir hipótesis sobre qué tan impactante puede resultar para la
economía de un individuo consumir una pizza con el agregado de su
preferencia.

\hypertarget{-cebolla}{%
\paragraph{-Cebolla:}\label{-cebolla}}

Sobre la cebolla, esta fue encontrada únicamente en agros en el
municipio, y era vendida únicamente por libra. Sin embargo, el margen de
precio del producto es muy amplio y como curiosidad, es más caro obtener
una libra de cebolla que una pizza. Esto será argumentado y analizado
próximamente haciendo uso de los precios y los tipos de cebolla que se
venden en Plaza de la Revolución.

\begin{quote}
\emph{"La cerveza, que tomada con exceso mata, como mata el exceso de
comida, reune todas las condiciones estimulantes que hacen deseables y
gustosas las bebidas alcohólicas, y no tiene ninguna de sus condiciones
destructoras.''}
\end{quote}

\begin{verbatim}
                                    >Jose Marti, La Opinión Nacional, 23 de febrero de 1882.
\end{verbatim}

Resultan interesantes las características que otorga El Apóstol a la
cerveza, bebida que arribó a nuestro país por primera vez en el siglo
XVII, de contrabando desde Jamaica. Fue en 1772 que la cerveza entra a
Cuba de manera legal y, naturalmente, como resultado del gusto por el
brebaje, el emprendimiento cubano comenzó a valorar la opción de crear
una cerveza nacional. Así, en 1841, Juan Manuel Asbert y Calixto García
comenzaron los primeros experimentos para producir la bebida en Cuba
reemplazando la cebada por la caña, lo cual fue un rotundo fracaso y dio
paso a que los primeros compases de la bebida en la colonia estuvieran
marcados por un dominio del producto originario de Inglaterra, que era
servido en tabernas y establecimientos similares, e hizo del
embotellamiento un lucrativo negocio alternativo. De las alrededor de
130 marcas que era posible consumir, la Tennent Lager gozaba de mayor
popularidad. Sin embargo, las barreras arancelarias impuestas por la
metrópoli no demoraron en hacer que producir la cerveza en el país fuera
más factible que solamente embotellarla y en 1888 se instaló en Puentes
Grandes la primera productora de cerveza cubana: "La Tropical".

A medida que se pulía la fórmula cervecera antillana, otras marcas
fueron apareciendo e incluso trascendieron más allá del territorio
nacional, ganando premios como el Diploma y Gran Premio en la Exposición
Internacional del Progreso, celebrado en París en 1912. Era tanta la
calidad de la que presumía la cerveza cubana, que muchos nacionales las
preferían sobre las ofertas extranjeras. Entre las marcas más populares
en nuestro país han estado por muchos años la Cristal, la Bucanero, la
Tinima, la Mayabe y la Cacique.

A partir de este breve recorrido histórico es posible realizar un
análisis de la situación de la cerveza en nuestro país, bebida que, como
ya se ha comentado, ha visto su auge y desarrollo verse obstaculizado o
incentivado a partir de medidas de índole económico pues, a nivel
social, aún goza de una popularidad extrema. A continuación, se
discutirán los resultados obtenidos a partir de una muestra de
establecimientos donde se vende esta bebida en el municipio Plaza de la
Revolución en La Habana.

\hypertarget{que-marcas-de-cerveza-abundan-mas-en-el-mercado-actual}{%
\paragraph{¿Que marcas de cerveza abundan mas en el mercado
actual?}\label{que-marcas-de-cerveza-abundan-mas-en-el-mercado-actual}}

\begin{verbatim}
Text(1.031, 0.7, 'Media de Apariciones por Cerveza: \n2.56')
\end{verbatim}

\includegraphics{deee570926c7529bc70950b25003e8c7ec8d10a0.png}

Sobre las demás, resalta la cerveza "Hollandia Premium", originaria de
Países Bajos, que se encontraba en 8 de los establecimientos analizados.
Con mayor frecuencia, también se encontraron en un segundo puesto la
"Victoria Málaga" y la "Windmill", seguidas de la Presidente.

Como es posible observar, la cantidad de establecimientos en los que
están presentes las bebidas nacionales no es positiva respecto a las
bebidas importadas. La mayor representación se encontró en la cerveza
Cristal, que fue localizada en 3 establecimientos del municipio
capitalino. Esto la sitúa por encima de la media respecto a la
disponibilidad del resto de marcas, la cual es de 2,56. Sin embargo, un
total de 25 marcas distintas que se encontraron, solamente tres eran
cubanas, lo que representa el doce por ciento del total de marcas
disponibles en las ubicaciones que se analizaron, que fueron 32. De
estos establecimientos, solamente en tres se encontraban disponibles
marcas nacionales, lo que representa el 9,4 por ciento del total de
establecimientos.

\includegraphics{3e2c758b1a67994b7b418fa72deb18dd71c37d3c.png}

\includegraphics{a1bfa372feeec458510dbfab59246017e21cbc59.png}

\hypertarget{como-se-distribuyen-los-precios}{%
\paragraph{¿Como se Distribuyen los
Precios?}\label{como-se-distribuyen-los-precios}}

La baja disponibilidad de cerveza nacional en el mercado minorista
responde a varios factores. La falta de insumos y de materias primas, la
falta de mantenimiento en las plantas procesadoras, dificultades
logísticas y una estructuración que prioriza el abastecimiento de otros
sectores son algunas de las causas de la escasez del producto. Este
desabastecimiento pasa entonces a ser un factor primordial para analizar
el precio de la bebida elaborada en Cuba.

En general, el costo promedio de la cerveza en el municipio Plaza de la
Revolución es de 190,55 pesos. La cerveza más cara, con un costo de 350
pesos, es la Corona, y la más barata, con un precio de 130,00 pesos, es
la Cruzcampo. Cabe destacar que dicha cerveza era ofertada en cajas de
24 unidades, por un costo total de 3,120.00 pesos. Sin embargo, el mismo
establecimiento comercializaba esta misma marca en 140,00 pesos, lo que
asegura su posición como la marca más asequible. La Hollandia Premium,
que es la marca más común, encuentra su precio topado a 250,00 pesos,
pero una lata puede ser adquirida por 170,00 pesos. Su precio medio es
de 200,00 pesos y su costo más común es también de 170,00 pesos.

En cuanto a la cerveza producida en Cuba, Cristal Extra se corona como
la más cara, con un precio máximo de 290,00 pesos, un precio mínimo de
220,00 pesos y un valor medio de 255,00 pesos. Le sigue la Cristal, con
un precio máximo de 280,00 pesos, y con el menor precio máximo se vende
la Bucanero, a 270,00 pesos.

La cerveza cubana no solo es menos abundante que la extranjera, sino que
también es más cara que la mayoría de las otras marcas. El precio mínimo
de la cerveza Cristal, que es la más barata de Cuba, está 9,45 pesos por
encima de la media del precio de la cerveza en el municipio. Solamente
la Corona supera en precio máximo a la Cristal Extra. Sin embargo, es
destacable también que la cerveza Corona se vende envasada en botella,
por lo que la Cristal Extra es la cerveza enlatada más cara entre las
marcas analizadas.


\includegraphics{6eb18331aec81b59f7cecd25eb0b96cce7304dda.png}

\includegraphics{52d9c7581eaa6230a44a74d47aea89032988a3a5.png}

Analizando la capacidad de compra del peso cubano por mililitro de
cerveza, nos encontramos con que casi todas presentan costo por
mililitro menor a un peso, excepto la Corona, que iguala esta cifra. La
más barata en este aspecto resulta la cerveza dispensada, seguida de la
Cruzcampo. Las cervezas nacionales siguen una tendencia similar al resto
de marcas y se reparten su lugar en la zona más poblada del gráfico.


\includegraphics{5a55922b7438d203af2efe6479208f655660729b.png}

Durante el análisis, también se consideró la cercanía de distintos
establecimientos entre sí. Se tomó la hipótesis de que un grupo de
establecimientos que estuvieran relativamente cerca tendrían precios
similares respecto a las mismas marcas de cerveza.

\includegraphics{0e16472896a80467788b30a96ef27e8cfc532a74.png}


\begin{verbatim}
grupo_de_cercania         1      2      3      4      5      6      7      8    
marca                                                                           
Bavaria                    -      -      -      -      -      -      -      -  \
Bergedorf                  -      -      -      -      -  180.0      -      -   
Bucanero                   -  250.0      -      -      -  270.0      -      -   
Claro                      -      -      -      -      -      -      -      -   
Cody's                     -      -      -      -      -  250.0      -      -   
Corona                     -      -      -      -      -  350.0      -      -   
Cristal                    -      -      -      -  250.0  250.0      -      -   
Cristal Extra              -      -      -      -  220.0      -      -      -   
Cruzcampo                  -      -      -      -      -      -      -      -   
Dispensada                 -      -      -      -      -      -      -  150.0   
Eichbaum                   -      -      -      -      -  175.0      -  180.0   
Germania                   -      -      -      -      -  155.0      -  135.0   
Hollandia Import           -      -      -  170.0      -      -      -  172.5   
Hollandia Premium      170.0  250.0  170.0      -      -  240.0      -  180.0   
Mahou                      -      -  180.0      -      -      -      -      -   
Martens                    -      -      -      -      -  150.0      -      -   
Naparbier Paradise     165.0      -      -      -      -      -      -      -   
Perlenbacher Patronus      -      -      -      -      -  250.0      -  200.0   
Presidente                 -  180.0  180.0      -      -  170.0  180.0  145.0   
Santa Isabel               -      -      -  180.0      -  160.0      -      -   
Shekels                    -  150.0      -      -      -  180.0  180.0      -   
Stella Artois          170.0      -      -      -      -  220.0      -      -   
Three Horses               -      -      -  200.0      -      -      -      -   
Victoria Malaga            -      -  180.0  150.0  160.0  200.0      -      -   
Windmill                   -      -      -  165.0  195.0  200.0      -  150.0   

grupo_de_cercania         9      10  
marca                                
Bavaria                    -  200.0  
Bergedorf                  -      -  
Bucanero                   -      -  
Claro                      -  260.0  
Cody's                     -      -  
Corona                     -      -  
Cristal                    -      -  
Cristal Extra              -  290.0  
Cruzcampo              135.0      -  
Dispensada                 -      -  
Eichbaum                   -      -  
Germania                   -      -  
Hollandia Import           -      -  
Hollandia Premium          -  180.0  
Mahou                      -      -  
Martens                    -      -  
Naparbier Paradise         -      -  
Perlenbacher Patronus      -      -  
Presidente                 -  130.0  
Santa Isabel               -      -  
Shekels                    -      -  
Stella Artois              -      -  
Three Horses               -      -  
Victoria Malaga            -  150.0  
Windmill                   -      -  
\end{verbatim}

Finalmente, fue posible comprobar que el precio de la cerveza resulta
muy variable incluso en establecimientos cercanos entre sí, y que en
general la cercanía entre dos establecimientos no garantiza que tengan
precios similares para su producto.


\begin{verbatim}
Grupo de Cercania 1:

 Hollandia Premium: 170
 Stella Artois: 170
 Naparbier Paradise: 165

Grupo de Cercania 2:

 Bucanero: 250
 Shekels: 150
 Presidente: 180
 Hollandia Premium: 250

Grupo de Cercania 3:

 Victoria Malaga: 180
 Hollandia Premium: 170
 Presidente: 180
 Mahou: 180

Grupo de Cercania 4:

 Three Horses: 200
 Santa Isabel: 180
 Victoria Malaga: 150
 Hollandia Import: 170
 Windmill: 180, 150

Grupo de Cercania 5:

 Windmill: 200, 190
 Victoria Malaga: 160
 Cristal: 280, 220
 Cristal Extra: 220

Grupo de Cercania 6:

 Santa Isabel: 160
 Germania: 155
 Windmill: 220, 180
 Stella Artois: 220
 Hollandia Premium: 230, 250
 Cristal: 250
 Bucanero: 270
 Victoria Malaga: 200
 Bergedorf: 180
 Perlenbacher Patronus: 250
 Eichbaum: 170, 180
 Presidente: 170
 Martens: 150
 Shekels: 180
 Corona: 350
 Cody's: 250

Grupo de Cercania 7:

 Presidente: 180
 Shekels: 180

Grupo de Cercania 8:

 Germania: 135
 Hollandia Import: 145, 200
 Presidente: 145
 Windmill: 150
 Perlenbacher Patronus: 200
 Hollandia Premium: 180
 Eichbaum: 180
 Dispensada: 150

Grupo de Cercania 9:

 Cruzcampo: 130, 140

Grupo de Cercania 10:

 Presidente: 130
 Cristal Extra: 290
 Bavaria: 200
 Hollandia Premium: 180
 Claro: 260
 Victoria Malaga: 150

\end{verbatim}

Según la Oficina Nacional de Estadística e Información, en el año 2022
el salario promedio en La Habana era de 4689,00 pesos. Tomando en cuenta
el precio medio de la cerveza en Plaza de la Revolución, nos encontramos
con que con el salario de un mes es posible adquirir una media de 24
cervezas. A continuación, una tabla muestra cuántas cervezas de cada
marca se pueden adquirir con el salario mensual en La Habana.


\begin{verbatim}
                    marca  precio_medio  cantidad
0                 Bavaria        200.00     23.44
1               Bergedorf        180.00     26.05
2                Bucanero        260.00     18.03
3                   Claro        260.00     18.03
4                  Cody's        250.00     18.76
5                  Corona        350.00     13.40
6                 Cristal        250.00     18.76
7           Cristal Extra        255.00     18.39
8               Cruzcampo        135.00     34.73
9              Dispensada        150.00     31.26
10               Eichbaum        176.67     26.54
11               Germania        145.00     32.34
12       Hollandia Import        171.67     27.31
13      Hollandia Premium        200.00     23.44
14                  Mahou        180.00     26.05
15                Martens        150.00     31.26
16     Naparbier Paradise        165.00     28.42
17  Perlenbacher Patronus        225.00     20.84
18             Presidente        164.17     28.56
19           Santa Isabel        170.00     27.58
20                Shekels        170.00     27.58
21          Stella Artois        195.00     24.05
22           Three Horses        200.00     23.44
23        Victoria Malaga        168.57     27.82
24               Windmill        181.43     25.84
\end{verbatim}

Respecto a los envases, la gama cromática de estos se divide en siete
colores, entre los que predominan el blanco y el verde, que juntos
constituyen el 82,8 por ciento del total de envases analizados. La lata
de cerveza tradicionalmente ha sido blanca o verde por varias razones,
entre ellas:

\begin{itemize}
\tightlist
\item
  Conservación de la temperatura: El color blanco y el verde claro
  reflejan la luz solar y reducen la absorción de calor. Esto ayuda a
  mantener la cerveza fría por más tiempo en un ambiente cálido.
\item
  Marketing y branding: Los colores blanco y verde claro se asocian a
  menudo con la frescura, la pureza y la naturaleza, lo cual puede ser
  beneficioso para la imagen de la marca de cerveza.
\end{itemize}

Sin embargo, es posible encontrar marcas de colores diversos incluso en
el escaso mercado nacional: la Cristal es presentada tradicionalmente en
un envase verde, la Cristal Extra en uno blanco y la Bucanero en uno
rojo. Esta diferenciación ha provocado que el cubano identifique
fácilmente el producto y que se cree un símbolo de identidad nacional a
partir de los colores que representan a las marcas autóctonas.

\includegraphics{908b6d4fc088fc157dfccdfd84abfc28a90c0ccb.png}

Otro aspecto relevante es la capacidad de dichos envases. La forma de
almacenamiento que más abunda es la lata, seguida de la botella, y hubo
un único establecimiento que servía cerveza dispensada. La capacidad más
común de los envases era de 330 ml, tanto para las bebidas nacionales
como extranjeras. En general, el color o la presentación del envase no
es posible afirmar que sean motivos de peso sobre el precio o la
disponibilidad del producto en el contexto económico-social actual, a
pesar de que dos colores dominan este apartado. Esto se debe a que los
importadores de cerveza no toman en cuenta las estrategias de marketing
detrás de la elección de color del envase para garantizar una mayor
venta, porque el peso de dicho apartado sobre la venta es mínimo.

Respecto a la cantidad, en general es posible afirmar que un mayor
envase implica un mayor precio del producto, excepto en el caso de la
cerveza dispensada que por 200,00 pesos proporcionaba 500 ml del
producto, convirtiéndola en la opción de mejor relación cantidad-precio
disponible. La cerveza embotellada es también más cara que su
contraparte enlatada.

\includegraphics{0cd7f0c9fc3672eb4e88e1f361584c13f2b3f68c.png}


\includegraphics{f9b9a0b73c2d75ba1bf1a8dc572503ea743ad5f4.png}

\hypertarget{de-que-paises-es-originaria-la-cerveza-disponible-en-el-municipio}{%
\paragraph{¿De que Paises es Originaria la Cerveza Disponible en el
Municipio?}\label{de-que-paises-es-originaria-la-cerveza-disponible-en-el-municipio}}

El país de origen más común para las cervezas resultó Países Bajos, con
cinco marcas presentes en los distintos establecimientos, lo que
representa el 20 por ciento del total de marcas. Le siguen España con 4
y Alemania, Bélgica y Cuba con 3 representantes cada uno. El origen de
la cerveza viene marcado por factores como: el costo del producto en el
país desde el cual se importa, el costo de los envíos y los
contenedores, la popularidad de la marca en la isla y la facilidad que
brinde el país desde el cual se importa para realizar la operación.


\includegraphics{84de827262b1f5416711fe439c132bc5b7d37166.png}

En general, el precio medio de la cerveza cubana es el más alto, junto a
la de Nueva Zelanda. Sin embargo, la bebida originaria del país asiático
únicamente contaba con un representante entre todas las marcas, como se
observó en el Treemap. Resulta interesante que España, con cuatro marcas
diferentes, tenga un costo medio de la bebida inferior a la media en el
municipio. Esto podría deberse a un menor costo de importación o más
facilidades para completar dicho proceso respecto a otros países.


\includegraphics{92736938d0d355c12f25ef2b09179221bde8148c.png}

\hypertarget{analizando-la-pizza}{%
\section{Analizando la Pizza}\label{analizando-la-pizza}}

Aunque muchos ubican el origen de la pizza en Cuba en los años 60, lo
cierto es que ya desde el XIX en la Isla comienza a conocerse la cocina
italiana, especialmente entre los de la burguesía criolla, según Ciro
Bianchi. Apunta el propio Bianchi que entre los años 40 y 50 se deja ver
algún que otro restaurante de comida italiana que ofrecían en su carta
la afamada pizza, entre ellos, el célebre Frascati, en Neptuno y Prado;
Doña Rosina o Montecatini en el Vedado. Pero indudablemente, fue en los
60 cuando las pizzas cubanizadas alcanzaron su verdadera popularidad;
unos dicen que por influencia de la moda italiana de aquella época,
otros afirman que se hicieron famosas porque eran un alimento barato y
rápido en ese entonces. A partir de los años 60 esto se convirtió en un
fenómeno y empezaron a surgir pizzerías por doquier en cualquier rincón
de la capital cubana.

Lo cierto es que hoy en día la rapidez con la que es posible obtener el
alimento y su precio relativamente barato respecto a otros platos han
hecho de la pizza una instancia de comida rápida extremadamente popular.
Precisamente son las características físicas de dicho producto las que
logranese cometido: son pizzas de tamaño mediano que pueden ser tomadas
con una sola mano y una masa suave que permite doblarla a la mitad sin
romperla y comerla mientras se camina o se está de pie sin sentir
especial incomodidad. También se caracteriza por la utilización de queso
Mozzarella, Blanco o Gouda; aunque las condiciones económicas del
productor y la disponibilidad de materias primas son las que
históricamente han condicionado los ingredientes utilizados en la
elaboración.

La pizza base analizada fue la Napolitana, únicamente de queso y
generalmente la versión más barata en un establecimiento determinado.
Cuando se mencione pizza de forma genérica se referirá a este tipo en
particular. Esta pizza generalmente es anunciada precisamente como
"pizza", y los agregos se suelen colocar aparte con sus precios
particulares. En algunos establecimientos, sin embargo, existen lo que
se denominó "Tipos Especiales". Esa característica engloba a los
productos con un nombre que no son reflejo de los agregos que tiene a
menos que así lo especifique el establecimiento. Por ejemplo: Si se
tiene una "Pizza Rocío" y una "Pizza Rocío de Jamón", y el precio de la
"Pizza Rocío de Jamón" es distinto al de la "Pizza Rocío"con agregado de
jamón, entonces ambas contarán como dos tipos especiales distintos.

\hypertarget{precios-de-la-pizza-en-plaza-de-la-revolucion}{%
\subsubsection{Precios de la Pizza en Plaza de la
Revolucion}\label{precios-de-la-pizza-en-plaza-de-la-revolucion}}

El precio de la pizza en el municipio se ve marcado fuertemente por
varios factores. Entre ellos, uno de los que más fuerza tiene es el
tamaño. La pizza tiene generalmente una forma circular, en donde un
mayor tamaño implica un mayor radio, pero no precisamente una mayor
altura o una mayor consistencia. A la pizza genérica le fue otorgado un
tamaño mediano. Esto permite definir si el producto que se está viendo
es mayor, menor o tiene una forma distinta a la tradicional.


\includegraphics{20874d2ed0acfc14e372f0d5ae4dbc2940e8af31.png}

Como se puede observar, el precio mínimo de una pizza grande es incluso
mayor que el precio máximo de una pizza mediana. El precio medio de la
pizza en Plaza de la Revolución es de 188,68 pesos cubanos. Esta media,
sin embargo, no es característica del precio habitual de la pizza, pues
existe una diferenciación muy grande entre los costos de las pizzas
grandes y las medianas. Sin embargo, la pizza mediana es mucho más
abundante que la grande y, por lo tanto, más fácil y común de encontrar
en los distintos establecimientos.


\begin{verbatim}
                   Moda  Precio Medio  Frecuencia Absoluta
tamano                                                    
Cono              160.0        160.00                    1
Grande   [240.0, 280.0]        305.56                    9
Mediana           150.0        158.08                   26
Pizzeta    [70.0, 80.0]         75.00                    2
\end{verbatim}

\hypertarget{zoom-in-a-la-pizza-grande-y-tipos-especiales}{%
\subsubsection{Zoom In a la Pizza Grande y Tipos
Especiales}\label{zoom-in-a-la-pizza-grande-y-tipos-especiales}}

Como se muestra, la pizza grande tiene un precio máximo de 400,00 pesos
y un precio mínimo de 260,00 pesos. Su precio medio es de 305,56 pesos,
lo que lo sitúa por encima del precio medio de la pizza en general.
Dentro de esta categoría de pizzas entran muchas de las consideradas de
Tipos Especiales, pues su mayor tamaño permite construir una oferta más
atractiva para el cliente a partir de asumir que ya gastará una cantidad
mayor de dinero. Estos tipos especiales tienen un precio muy variado y
generalmente son platos característicos del lugar donde son servidas. La
pizza Hawaiana es la pizza de tipo especial más común, encontrándose en
6 de los establecimientos analizados, y la oferta más cara fue una Pizza
Doble Queso, con un precio de mil cuatrocientos pesos cubanos en dos
establecimientos diferentes.


\begin{verbatim}
                     tipos_especiales             precio_especiales_x   
0                         Doble Queso                          [1400]  \
1                            Especial                      [230, 400]   
2                   Especial de Queso                      [400, 320]   
3             Especial de Queso Gouda                           [400]   
4                           Hawaiiana  [360, 400, 520, 280, 270, 260]   
5                               Mixta                           [300]   
6      Pizza Extrafina de Queso Gouda                           [300]   
7                         Pizza Rocio                          [1000]   
8                   Pizza Tradicional                           [190]   
9   Pizza Tradicional con Queso Gouda                           [230]   
10               Pizza de Queso Gouda       [400, 200, 250, 500, 200]   

    precio_especiales_y  
0                     1  
1                     2  
2                     2  
3                     1  
4                     6  
5                     1  
6                     1  
7                     1  
8                     1  
9                     1  
10                    5  
\end{verbatim}

Del total de ubicaciones analizadas, el 34.6 por ciento contaba con
ofertas de pizzas especiales, lo que demuestra una capacidad moderada de
los establecimientos para tener esta característica única sobre los
demás.


\includegraphics{72fdedf53d4fadb1faae932bd2074749981d2664.png}


\includegraphics{1a4bf0512d119b88c879599fc2821962cdf299f5.png}

Como se observa, el agrego más caro resulta el Jamón Serrano, y el
precio medio para los agregos de las pizzas de tamaño grande es de 121,1
pesos. Los agregos más baratos coinciden con productos de producción
agrícola nacional: la cebolla, el pimiento, la piña y el tomate. La
disponibilidad de estos productos y la facilidad de adquisición respecto
a los productos importados los hacen una constante en casi todos los
establecimientos. El precio medio de la pizza grande en Plaza de la
Revolución es de 305,56 pesos y no hay una moda en los precios. En
general, la pizza grande constituye un producto presente en una menor
proporción en los establecimientos analizados.

\hypertarget{zoom-in-a-la-pizza-mediana}{%
\subsubsection{Zoom In a la Pizza
Mediana}\label{zoom-in-a-la-pizza-mediana}}

La pizza mediana es el tipo más común en los establecimientos no
estatales de venta minorista que producen este producto. Cuando se
piensa en la pizza cubana, esta es la que engloba las características
más esenciales. Este tipo de pizza tiene un precio promedio de 158,08
pesos y está presente en la amplia mayoría de los establecimientos que
se analizaron.

Para la pizza mediana, el precio medio de un agregado es de 77,6 pesos,
pero es notable cómo el precio máximo de cada agregado está por encima
de la media. Este fenómeno encuentra una explicación en la moda de los
distintos precios de los agregados que había disponibles. La moda de los
precios de la pizza mediana es de 150,00 pesos.


\includegraphics{c1580c65dbb48fa978bebe5073a128235465c8c3.png}

\includegraphics{0cff918cb7db278ecb642a610e2528cc8fcb6e90.png}

\hypertarget{pizza-mediana-y-grande}{%
\subsubsection{Pizza Mediana y Grande}\label{pizza-mediana-y-grande}}

A partir de los datos procesados sobre las pizzas medianas y grandes, se
puede establecer la siguiente relación sobre la presencia de agregados
en cada uno de los tipos de pizza y al mismo tiempo hacer una
comparación sobre sus respectivos precios. El gráfico se refiere al
precio máximo en cada tipo de pizza, siendo el color rojo el precio
máximo del agregado en la pizza grande y el morado en la pizza mediana.
El azul indica el valor máximo en el que alcanzan el mismo precio, si es
que lo alcanzan.


\includegraphics{8dece6e4b6e675bfddb20bed0dfe5459338b0894.png}

Sobre el precio de la pizza, también se manejó la hipótesis de que el
costo del producto en ubicaciones cercanas a puntos de interés, como
escuelas u hospitales, podría incrementarse respecto a las ubicaciones
que no tienen un punto de interés cercano.


\includegraphics{d2ae00ccdc2f4832c75602bc0e98921a2425e10b.png}

Sin embargo, los datos muestran que los precios en establecimientos con
lugares de interés cercanos no solo son menores en general que los que
no tienen, sino que además el precio es menor a la media del costo medio
de la pizza en el municipio. Esto podría deberse a que dentro de una
zona cercana a un lugar de interés suele haber más establecimientos
cercanos entre sí, lo que podría provocar un aumento de la competencia
y, a su vez, llevar a precios más competitivos entre los puntos de venta
dentro de un mismo grupo de cercanía.


\includegraphics{a6fd7a73e2fdb65a6209456c107bb6b839de8bc9.png}

Precisamente, el grupo de cercanía 3 se encuentra en los alrededores de
la Universidad de La Habana y el Hospital "Calixto García". Y el grupo
de cercanía 6 contiene ubicaciones cercanas al Hospital "González Coro".

\hypertarget{relacion-pizza---cerveza---salario}{%
\subsubsection{Relacion Pizza - Cerveza -
Salario}\label{relacion-pizza---cerveza---salario}}

Con el Salario Mensual de La Habana de 4689,00 pesos, tomando como
referencia la moda del precio de la cerveza y la moda del precio de las
pizzas, es posible conformar el precio de lo que podría ser un almuerzo
para una persona en puntos de venta donde se distribuyan estos
productos. Nótese que se toma la moda porque es el precio más común en
el que se pueden encontrar ambos productos. Así, la moda del precio de
la pizza es de 150,00 pesos, y la de la cerveza es de 180,00 pesos, por
lo que un almuerzo de una pizza y una cerveza es muy probable que cueste
alrededor de 330,00 pesos. Esto implica que con el salario medio se
pueden comprar 14 almuerzos si se utiliza completamente para ello.

Este número resulta pequeño al considerar el resto de necesidades
básicas de una persona e incluso al considerar el alto costo que tienen
algunas de las cervezas y pizzas que formaron parte del análisis, lo que
evidencia la pobreza de la capacidad de compra del peso cubano en el
contexto actual, donde ni el producto nacional es capaz de salvar la
situación al verse dominado en cuanto a la competitividad de los precios
por el producto importado.

\hypertarget{analizando-la-cebolla}{%
\section{Analizando la Cebolla}\label{analizando-la-cebolla}}

La cebolla en Cuba se vende principalmente en agros o carretillas que
recorren la ciudad y promocionan el producto, el tamaño, el tipo de
cebolla e incluso, a veces, es posible escuchar el precio si se agudiza
el oído hacia cualquier conversación entre el vendedor y cualquier
curioso.

En general, la cebolla más abundante se divide en dos tipos: morada y
blanca, aunque hay otros tipos que son menos comunes en los distintos
establecimientos, como la Caribe.


\begin{verbatim}
  tipo de cebolla  precio_x  precio_y
0          Blanco    187.65       170
1          Caribe    150.00       150
2          Morada    187.06       170
\end{verbatim}


\begin{verbatim}
El valor medio de la cebolla en Cuba es de 170.38 pesos
\end{verbatim}


\includegraphics{86a9d801a4ecd6810002dd0c64e06cf70ac6ee0c.png}

Como se puede observar, la cebolla blanca tiene un precio medio muy
similar al de la morada, sin embargo, la cebolla morada tiene un precio
máximo mayor, llegando a costar 260,00 pesos cubanos.

Si consideramos que a una pizza mediana se le agregan 150 gramos de
cebolla, lo que equivale a 0,33 libras, entonces de una libra de cebolla
se pueden completar 3 pizzas. La cebolla más común con la que se realiza
la pizza es la blanca, cuyo precio modal es 170,00 pesos. Entonces, de
una libra de cebolla, 56,6 pesos irían a cada una de las pizzas. Esto
situaría el costo del agregado de cebolla por encima de la moda del
agregado de cebolla para pizzas medianas, que es de 40, por lo que no
resulta una suposición descabellada que los pizzeros puedan obtener la
cebolla más barata que el público general o que conceban las pizzas con
menos de 150 gramos de cebolla, lo cual resulta una cantidad irrisoria.

En cuanto al salario provincial y la cebolla, resulta que el precio
medio de la cebolla en el municipio es de 170,38 pesos, por lo que con
el salario de un mes sepodrían comprar 27 libras de cebolla. Esto es
otra evidencia de la inflación y la pérdida de valor de la moneda
nacional respecto incluso a productos propios.
